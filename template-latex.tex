% Template LaTeX pour colorer les titres
\usepackage{xcolor}

% Définition des couleurs
\definecolor{primaryblue}{RGB}{37,99,235}  % Bleu principal
\definecolor{secondaryblue}{RGB}{59,130,246}  % Bleu secondaire
\definecolor{darkblue}{RGB}{29,78,216}  % Bleu foncé

% Redéfinition des commandes de section pour ajouter des couleurs
\makeatletter
\renewcommand{\@seccntformat}[1]{%
  \textcolor{primaryblue}{\csname the#1\endcsname\quad}%
}

% Titre de niveau 1 (h1) - Bleu foncé
\let\old@maketitle\@maketitle
\renewcommand{\@maketitle}{%
  \old@maketitle
  \let\@maketitle\old@maketitle
}

% Section (h2) - Bleu principal
\renewcommand{\section}{%
  \@startsection{section}{1}{\z@}%
  {-3.5ex \@plus -1ex \@minus -.2ex}%
  {2.3ex \@plus.2ex}%
  {\normalfont\Large\bfseries\color{primaryblue}}%
}

% Subsection (h3) - Bleu secondaire
\renewcommand{\subsection}{%
  \@startsection{subsection}{2}{\z@}%
  {-3.25ex\@plus -1ex \@minus -.2ex}%
  {1.5ex \@plus .2ex}%
  {\normalfont\large\bfseries\color{secondaryblue}}%
}

% Subsubsection (h4) - Bleu secondaire plus clair
\renewcommand{\subsubsection}{%
  \@startsection{subsubsection}{3}{\z@}%
  {-3.25ex\@plus -1ex \@minus -.2ex}%
  {1.5ex \@plus .2ex}%
  {\normalfont\normalsize\bfseries\color{secondaryblue}}%
}

\makeatother

% Pour pandoc, on utilise une approche différente
% On redéfinit les commandes de section après le préambule
\AtBeginDocument{%
  \let\oldsection\section
  \renewcommand{\section}[1]{%
    \oldsection{\textcolor{primaryblue}{#1}}%
  }%
  \let\oldsubsection\subsection
  \renewcommand{\subsection}[1]{%
    \oldsubsection{\textcolor{secondaryblue}{#1}}%
  }%
  \let\oldsubsubsection\subsubsection
  \renewcommand{\subsubsection}[1]{%
    \oldsubsubsection{\textcolor{secondaryblue}{#1}}%
  }%
}

